\chapter{Working in Hindi}
\label{chap:hindi}

Devanagari script is the script for writing Hindi language. Hindi is the official language of India. Hindi is spoken in almost all of India. It includes 13 vowels and 36 consonants. Apart from this, it has basic 11 modifiers which are combined with different consonants and vowels.

Each vowel except the first one has a corresponding modifier using which we can modify a consonant. This line which is available on the upper side of a character is called “Shirorekha''. Based on this shirorekha each character is divided into three distinct parts. The portion in the upper side of shirorekha is called upper modifiers, in the middle portion the character is available and in the last portion lower modifiers are available. 
\subsubsection{The major problems with this script which require special attention are:}
\begin{enumerate}
\item “Shirorekha” or the header line above each and every character.
\item Attachment of modifiers before, after, above, below and within the base vowels and consonants.
\item Large number of symbols
\item Joint, touching and broken characters
\end{enumerate}